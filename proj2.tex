\documentclass[a4paper, 11pt, twocolumn]{article}
\usepackage{graphicx} 
\usepackage[czech]{babel}
\usepackage[T1]{fontenc}
\usepackage[utf8]{inputenc}
\usepackage[left=1.4cm, top=2.3cm, text={18.3cm, 25.2cm}]{geometry}
\usepackage{lmodern}
\usepackage{amsthm, amssymb, amsmath}
% \usepackage{hyperref}

\theoremstyle{definition}
\newtheorem{definition}{Definice}

\begin{document}

    \begin{titlepage}
		\begin{center}
			{\Huge\textsc{
				\textsc{\Huge{Vysoké učení technické v Brně}\\[0.5em]
                \quad\huge{Fakulta informačních technologií}}
			}}
			\vspace{\stretch{0.381966}}
			{\LARGE
				\\ Typografie a~publikování \,--\ 2. projekt\\[0.6em]
				Sazba dokumentů a~matematických výrazů
			}
            
			\vspace{\stretch{0.618034}}
		\end{center}

		{\Large
			\the\year
			\hfill
			Petr Vitula (xvitulp00)
		}
	\end{titlepage}

    \section*{Úvod}

	V této úloze si vysázíme titulní stranu a kousek matematického textu, v němž se vyskytují například Definice \ref{definition_1} nebo rovnice \eqref{eq_2} na straně \pageref{definition_matematika}. Pro vytvoření
    těchto odkazů používáme kombinace příkazů \verb|\label|,
    \verb|\ref|, \verb|\eqref| a \verb|\pageref|. Před odkazy patří nezlomitelná mezera. Pro zvýrazňování textu se používají
    příkazy \verb|\verb| a \verb|\emph.|
    
    Titulní strana je vysázena prostředím \verb|titlepage|
    a~nadpis je v optickém středu s využitím \emph{přesného} zlatého řezu, který byl probrán na přednášce. Dále jsou
    na titulní straně čtyři různé velikosti písma a mezi
    dvojicemi řádků textu je použito řádkování se zadanou relativní velikostí 0,5\,em a 0,6\,em\footnote{Použijte správný typ mezery mezi číslem a jednotkou.}.
    
    \section{Matematický text}

	Matematické symboly a výrazy v plynulém textu jsou
    v prostředí \verb|math|. Definice a věty sázíme v prostředí
    definovaném příkazem \verb|\newtheorem| z balíku \verb|amsthm|.
    Tato prostředí obracejí význam \verb|\emph|: uvnitř textu
    sázeného kurzívou se zvýrazňuje písmem v základním řezu. Někdy je vhodné použít konstrukci \verb|${}$|
    nebo \verb|\mbox{}|, která zabrání zalomení (matematického) textu. Pozor také na tvar i sklon řeckých písmen:
    srovnejte \verb|\epsilon| a \verb|\varepsilon|, \verb|\Xi| a \verb|\varXi|.

    \begin{definition}
        \label{definition_1}
		Konečný přepisovací stroj neboli Mea\-lyho automat je definován jako uspořádaná pětice
    tvaru M = (\(Q, \Sigma, \Gamma, \delta, q_0\)), kde:
    \begin{itemize}
        \item Q je konečná množina stavů,
        \item $\Sigma$ je konečná vstupní abeceda,
        \item $\Gamma$ je konečná výstupní abeceda,
        \item $\delta : Q \times \Sigma \rightarrow Q \times \Gamma$ je totální přechodová funkce,
        \item $q_0 \in Q$ je počáteční stav.
    \end{itemize}
	\end{definition}

    \subsection{Podsekce s definicí}
    
    Pomocí přechodové funkce \(\delta\) zavedeme novou funkci~\(\delta^*\)
    pro překlad vstupních slov \(u \in \Sigma^*\) do výstupních slov
    \(w \in \Gamma^*\).

    \begin{definition}
		\label{definition_matematika}
        Nechť $M = (Q, \Sigma, \Gamma, \delta, q_0)$ je Mealyho automat. Překládací funkce $\delta^* : Q \times \Sigma^* \times \Gamma^* \rightarrow\; \Gamma^*\;$ je pro každý stav $q \in Q$, symbol $x \in \Sigma$, slova $u \in \Sigma^*$, \( w \in \Gamma^* \) definována rekurentním předpisem:
    \begin{itemize}
        \item $\delta^*(q, \varepsilon, w) = w$
        \item $\delta^*(q, xu, w) = \delta^*(q_0, u, wy)$, kde $(q_0, y) = \delta(q, x)$
    \end{itemize}
	\end{definition}

    \subsection{Rovnice}

    Složitější matematické formule sázíme mimo plynulý
    text pomocí prostředí \verb|displaymath|. Lze umístit i více
    výrazů na jeden řádek, ale pak je třeba tyto vhodně
    oddělit, například pomocí \verb|\quad|, při dostatku místa
    i~\verb|\qquad|.
    $$
		g^{a_n} \notin A^{B_n}
		\qquad
		y_0^1 - \sqrt[5]{x+\sqrt[7]{y}}
		\qquad
		x > y^2 \geq y^3
	$$

    Velikost závorek a svislých čar je potřeba přizpůsobit jejich obsahu. Velikost lze stanovit explicitně,
    anebo pomocí \verb|\left| a \verb|\right|. Kombinační čísla sázejte makrem \verb|\binom|.

    \begin{equation*}
    \left\lvert\bigcup P\right\rvert = \sum_{\emptyset \neq X \subseteq P} (-1)^{\lvert X \rvert-1} \ \left\lvert \bigcap X \right\rvert
    \end{equation*}
    
    \begin{equation*}
    F_{n+1} = \binom{n}{0} + \binom{n-1}{1}+\binom{n-2}{2}+\dots+\binom{\left\lceil\frac{n}{2}\right\rceil}{\left\lfloor\frac{n}{2}\right\rfloor}
    \end{equation*}

    \vspace{0.1cm}
    
    V rovnici \eqref{eq_1} jsou tři typy závorek s různou \emph{explicitně} definovanou velikostí. Obě rovnice mají svisle zarovnaná rovnítka. Použijte k tomu vhodné prostředí.
   \begin{align}
        \label{eq_1}
        \bigg( \Big\{ b \otimes \big[ c_1 \oplus c_2 \big] \circ a \Big\}^{\frac{2}{3}} \bigg) & \quad=\quad \log_z x \\
        \label{eq_2}
        \int_a^b f(x) \, \mathrm{d}x & \quad=\quad -\int_b^a f(y) \, \mathrm{d}y
    \end{align}

   V této větě vidíme, jak se vysází proměnná určující limitu v běžném textu: $ \lim_{m \rightarrow \infty} f(m) $. Podobně je to i s dalšími symboly jako $\bigcup_{N \in \mathcal{M}}$N či $\sum^m_{i=1}x^2_i$. S vynucením méně úsporné sazby příkazem \verb|\limits| budou vzorce vysázeny v podobě $\lim\limits_{m \rightarrow \infty} f(m)$ a $\sum\limits_{i=1}^{m} x_{i}^{2}$.


   \section{Matice}
    Pro sázení matic se používá prostředí array a závorky
    s výškou nastavenou pomocí \verb|\left|, \verb|\right|.
    \begin{equation*}
    D=\left|\begin{array}{cccc}
    a_{11} & a_{12} & \cdots & a_{1 n} \\
    a_{21} & a_{22} & \cdots & a_{2 n} \\
    \vdots & \vdots & \ddots & \vdots \\
    a_{m 1} & a_{m 2} & \cdots & a_{m n}
    \end{array}\right|=\left|\begin{array}{cc}
    x & y \\
    t & w
    \end{array}\right|=x w-y t
    \end{equation*}

    Prostředí \verb|array| lze úspěšně využít i jinde, například
    na pravé straně následující rovnosti.

    \begin{equation*}
        \binom{n}{k}= \begin{cases}\quad\frac{n !}{k !(n-k) !} & \text { pro } 0 \leq k \leq n \\
        \quad0 & \text { jinak }\end{cases}
    \end{equation*}


\end{document}

